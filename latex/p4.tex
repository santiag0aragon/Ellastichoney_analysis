Our main focus for the analytics will include identifying clear statistics on events from whom, from where and what attacks recorded in the JSON logs available. The following definitions will help us critically analyze the nature, criticality, severity and resources needed to mitigate the impact of this vulnerability.  
•	The rank order or the concentration from where the majority of attacks are coming from, considering the geographical locations / regions to be identified such as continent, country and city name. 
•	The number of events that took place during the detection phase and the time duration period / timestamp of the attack, this will demonstrate the extent of effort put in to gain access to critical systems.
•	The host systems that are under attack targeting specific ports.
•	The types of content used in the vulnerability.
•	The different methods and types of various user agents used to exploit the vulnerability paying particular attentions to the application names.
•	The flavor of operating systems used and the versions.
•	The form and payload strings matching the actual exploit.
•	The majority of source IPs sending data traffic.
•	The method and nature of data transfer via URLs for direct path to vulnerabilities that are exploitable.
•	The most important of all is the breakdown of the individual payload and its various instruction sets.
•	The segregation between valid and invalid vulnerabilities marking down malicious IPs and the percentage of false positives
•	The number of recurring attacks from particular IPs.
•	Understanding the status of a vulnerability – whether it is New, Active, Reopened, Verified, Excepted, Pending Remediation or Fixed
•	Bandwidth consumed during the attacks.

There are other exploitability metrics that can be derived from the logs such as:
•	Access Vector
•	Access Complexity
•	Authentication
•	Confidential Impact
•	Integrity Impact
•	Availability Impact
