\begin{itemize}
\item \textit{Vendors of Operating Systems and Softwares.} The market for lemons that Akerlof described explains that because buyers can’t distinguish the quality of high vs low, they refused to pay a premium price for high quality goods. Vendors of Operating Systems and software may encounter the same market features since the market for secure software is a market for lemons. Vendors may market and try to sell their software as secure but it is hard for an end user to understand and believe this, compared to the alternative cheaper options. Because of that, it will eventually have a lesser return on the vendors’ investment to mitigate the security issue, if buyers can’t be convinced that it is more secure and safe for their productive use. Buyers look at features that they can measure the quality of, such as user interface and price. So, vendors put more effort on satisfying the customer base through features and benefits that can actually be observed, that leads to a bad outcome because security is not emphasised as it should be.

\item \textit{End Users.} Users may change their behaviour if they are given some kind of protection to use the machines, which can cause a moral hazard because they will take security less seriously. User are actually paying for security software to mitigate the risk rather than actually solving the problem.


\item \textit{ISPs.} ISP’s profits can increase when they clean up botnet malware from the network which would result an increase in users’ demand for network(bot-free) access. At the same time, if the clean-up cost per user largely raises the access fee, the demand for network access and ISP’s profits can also decrease~\cite{asghari2010botnet}. Because detecting and cleaning up botnet malware in users’ computers are expensive, ISPs may rather choose to disconnect users whose computers are vulnerable to malware infection~\cite{kinukawa2012should}. Hence, some external funding in an effort to fight botnets would encourage ISPs. In this direction, a government-sponsored program have been helpful to ISPs in the past[example: Australia and Germany]. In the case governments are unwilling to fund these initiatives, ISPs have to find a way to make them, at the very least, cost neutral if not cost positive~\cite{asghari2010botnet}. Considering the increasing trend of botnet ad-fraud attacks and the consequently increasing loss of ad revenue for Ad networks, Ad Networks have economic incentives to fight botnets. However, Ad Networks are not in the best position to thwart botnets themselves and thus ANs might be willing to subsidize the ISPs to achieve that goal. Such cooperation would motivate and financially help ISPs to deploy detection and remediation mechanisms to help fight botnets~\cite{vratonjic2010isps}.

\end{itemize}

% Cloud Service Providers:
% The customers pay for the cloud services, there is(legal) objections to any attack(botnet) on their data and the price of Cloud services would continue to drop if bot activities on the Cloud affect users. Cloud service providers are legally  bound to protect legitimate users of Cloud services from botnet infection. Cloud Service Providers have economic incentives to fight botnets~\cite{peng2014moving}.

% End Users:

% Home users are well aware of the consequences of botnet infection due to their valuable data, money and productivity at stake, and some of them join arms in fighting botnets. It might also be that unlike ISP, these home users can directly enjoy the incentives of having protected machines. On the other hand, it is about security versus availability, speed, or usability in their business and finally, the monetary benefits of security measures not being very explicit might not motivate all the users to do the same~\cite{asghari2010botnet}.

% Mid and small sized businesses relying on online services suffer as they lack necessary resources to offer safe financial transactions for their customers and their bigger concern would be about their turnover and staying competitive, rather than investing on botnet mitigation~\cite{asghari2010botnet}.

