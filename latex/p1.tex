\label{cap:sec1}

It is impractical and very expensive to defend every possible vulnerability~\cite{TARA}. Therefore, decision-makers must plan effective security strategies about where and when to invest time and money~\cite{ROSI}. To make decisions on security investments, security investment models are used and these models are build on security metrics~\cite{BR2010}. Not all vulnerabilities will be attacked, so knowing the most likely attack vectors is key in planning effective security measurements~\cite{TARA}. For this knowledge, data about attacking behavior is needed. If we have this data, we can define metrics that measure different aspects of attacking behavior.

One way of getting data about attacking behavior is by using \textit{honeypots}. Honeypots are systems made vulnerable intentionally to attract attackers and track their behavior~\cite{WP2010}. An example of a honeypot is \textit{Elastichoney}~\cite{BR2010}. Elastichoney is an elasticsearch honeypot containing the RCE Remote code execution vulnerability -CVE- 2015-1427, which allowed attackers to execute java based code in its search bar~\cite{CVE}. The log files of Elastichoney contain data collected over two months and tracked about 8k attempts to attack from over 300 unique IP addresses~\cite{BR2010}.

When looking at the data collected by Elastichoney, a major security issue can be found, namely \textit{Botnet Infection}. The attacking behavior of botnet infection found in the data can be summarised as follows:
\begin{itemize}
\item[-] Using the code execution vulnerability, operation steps were performed on multiple different files: \textit{wget} file, \textit{chmod 777} on file location, execute file, removing file.   
\item[-] Operations on the same file were requested by multiple different sources.
\end{itemize}
In this paper we will look at two parties who have great incentive to know about the attacking behavior of botnet infection, namely \textit{Internet Service Providers (ISP's)} and \textit{Enforcement agencies}.

Botnet infection is a major security issue for ISP's, because 27\% of the overall unwanted traffic on the Internet can be attributed to botnet-related spreading activity~\cite{AM2006}. And this is a great loss in productivity for ISP's.  

Enforcement agencies want to counter botnet infection, because botnets are heavily used as a platform for various criminal business models like sending spam, committing click fraud, harvesting account credentials, launching denial-of-service attacks, installing scareware and phishing~\cite{AR2013}.

The rest of this paper will discuss metrics that measure different aspects of the attacking behavior of botnet infection and how these metrics can be used by ISP's and enforcement agencies in making effective security investments.

