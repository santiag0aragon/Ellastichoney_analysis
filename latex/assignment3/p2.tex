%!TEX root = main.tex
 ISPs:

 If ISPs do not take any measures, it would translate into bandwidth being eaten up by botnet activities which would urge them for investments in infrastructure expansion, and hence ISPs must consider investing on this issue. As all security comes at a cost, the ISPs choose their measures based on their mix of incentives and cost perception. The two involved in above stated mitigation efforts would be better detection and remediating the installations of malicious bots. The data feeds that the ISPs are currently using, does not give them adequate intelligence on the total number of infected machines in their network. There are additional data sets that ISPs can make use of to improve their intelligence, this is going to be expensive when each ISPs try to have one for itself. It is possible to invest less by building one platform for all ISPs, rather than each ISP building a platform on its own. A centralized, shared clearinghouse might be an efficient way to drastically improve the intelligence that ISPs are using to protect their networks and customers against modest cost [Ex: The Australian Communications and Media Authority (ACMA) has established a clearinghouse that aggregates numerous data feeds and transforms them into weekly reports for each Australian ISP]~\cite{asghari2010botnet}.

 The second option, improving the mitigation of infected machines, focuses on ways to enable ISPs to better deal with infected customers. Sharing tools and procedures would be helpful in here. The critical issue will be to reduce the cost of customer contact and support. The more efficient an ISP can deal with a customer, the more infections it can take action on, within the same amount of resources~\cite{asghari2010botnet}.

