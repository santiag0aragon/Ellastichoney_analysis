%!TEX root = main.tex
\label{cap:sec1}


Botnet infection is a serious threat for a number of entities: end users, businesses, websites, Internet Service Providers (ISPs), Cloud Service Providers and Law Enforcement Agencies. Consequently, thwarting botnets would benefit all these entities. However, the problem of botnet infection in general cannot be solved exclusively by any of these entities alone. In this direction, we propose that each entity should take measures to fight the botnet growth.

\subsection{ISPs}

1)Internet Service Providers can detect those systems that have been infected with malicious bots, notify the same to Internet users and use various remediation(remove, disable, or otherwise render a bot harmless) techniques. ISPs hold a unique position(DNSBLs, honeypots, darknets, passive DNS, traffic flow based and log analysis techniques) in fighting botnets because of  their role as provider of IP connectivity, which gives them the ability to act upon the bot traffic and their access to the end users (to contacting them, or asking them to install certain software, etc).Mitigating the effects of and remediating the installations of malicious bots will make it more difficult for botnets to operate and could reduce the level of botnet activities in general and/or on a particular Internet Service Provider's network~\cite{anderson2013measuring}. Industry collaborative efforts like the Internet Engineering Taskforce (IETF) and the Messaging Anti-Abuse Working Group (MAAWG) have prepared sets of best practices for the remediation of bots in ISP networks. ISPs can opt for Firewall and security policy changes at the network level, can go for Port 25 management, walled gardens to quarantine infected users, can filter Inbound and outbound email and distribute secure ICT infrastructure to users to mitigate botnet infection~\cite{charney2012collective}. 
	

2)ISPs are in the best position to detect the presence of a botnet and to take measures against it. ISPS can use technical means that can slow the botnet down. An example for this would be consuming its resources. ISPs can take these measures by performing Denial of Service attacks against Command-and Control Servers of the botnet, trapping and holding connections from infected machines, or blocking of malicious domains~\cite{leder2009proactive}. 

\subsection{Cloud Providers}

To have cloud infrastructure in which configuration constantly evolves to confuse attackers without significantly degrading the quality of service. Proposed solutions may increase the cost for potential attackers by complicating the attack process and limiting the exposure of network vulnerability in order to make the network more resilient against novel and persistent attacks. This can be done by using following technologies: Polymorphism- Develop the novel cloud defence polymorphism techniques to protect cloud infrastructures from attackers[Use botnet polymorphism techniques, i.e., server-side polymorphism and malware polymorphism], Agility- Develop the rapid provisioning technologies of cloud resources to provide high resource availability to cloud customers[Investigate botnet agility behaviours] and Poisoning Prevention- Develop the tamper evident technologies that make unauthorized access to the protected cloud resources easily detected[Probe botnet poisoning mechanisms]~\cite{peng2014moving}.

\subsection{End Users}

End-users comprise of individuals users and small to medium sized businesses. Home users have to start using antivirus and firewall software as part of personal botnet mitigation measures. Small to medium businesses should fight botnet in association with ISPs by coming up with various ‘best practice recommendations’ regarding Internet security. It is win-win for both the entities~\cite{asghari2010botnet}.

