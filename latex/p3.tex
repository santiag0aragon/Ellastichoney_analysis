%!TEX root = main.tex


In the previous section we have seen which kind of metrics would be ideal for ISP's and LEA's in facing the botnet infection issue.

Current research mostly use honeypots to detect where botnet infection traffic is coming from and how it grows over time. In these cases, metrics set out the number of connections/attacks over time to the different IP addresses/longitudinal locations/domain names~\cite{AM2006}.

Other research has been done in differentiating between botnet types by looking at the connection and resources features of different botnet types. After classification, metrics set out the number of connections/attacks over time to the different botnet types~\cite{GJ2007},~\cite{AM2005}.

And last, research has been done in training honeypots to automatically detect botnet infection traffic and metrics are used to show the percentage of correct detections/classifications for th different botnet types~\cite{haltas2014automated}.
Some examples of this metrics are mentioned here:
\paragraph{Cluster quality }

The  parameters such as total number of flows, Net flow size, internal host count, concurrently  active, start date , end date and length(days) of botnets are measured. The cluster quality is then analyzed by using accumulated botnet samples and traces ~\cite{haltas2014automated}.

Enforcement agencies use this metric to know about possible network attacks and botnet activities.

% The following  figure represent  cluster quality over time of botnets (Carberp, Hesperbot, TInba, Ramnit, Gamarue and Cridex) with details of measurement mentioned in the next figure.

\paragraph{ Detection rate }

The  parameters such as total number of flows, Net flow size, internal host count, concurrently  active, start date , end date and length(days) of botnets are measured. The detection rate is then analyzed by using accumulated botnet samples and traces ~\cite{haltas2014automated}.

Enforcement agencies use this metric to know about possible network attacks and botnet activities.

% The following  figure represent  cluster quality over time of botnets (Carberp, Hesperbot, TInba, Ramnit, Gamarue and Cridex) with details of measurement mentioned in the above figure 2.


\paragraph{ Infection rate }

The  parameters such as total number of flows, Net flow size, internal host count, concurrently  active, start date , end date and length(days) of botnets are measured. The detection rate is then analyzed by using accumulated botnet samples and traces ~\cite{haltas2014automated}.

Enforcement agencies use this metric to know about possible network attacks and botnet activities.

% The following  figure represent  cluster quality over time of botnets (Carberp, Hesperbot, TInba, Ramnit, Gamarue and Cridex) with details of measurement mentioned in the above figure 2.
\paragraph{ Connection ratio }

This is the ratio of number of bots in the largest connected graph to number of remaining bots (peer to peer botnets).

Enforcement agencies use this metric to know how well a botnet survives a defence action.

\paragraph{ Connection degree ratio }

This is the ratio of average degree of the largest connected graph to average degree of original botnet(peer to peer botnets).

Enforcement agencies use this metric to know how densely the remaining botnet is connected together after bot removal.

\paragraph{ Network diameter }

Network diameter is the average geodesic length of the network.  The dynamics of the network such as communication, information and epidemics are slow if "l" is large. This metric serves to be relevant  because  with every message passed through botnet , probability of failure or disconnection exists.~\cite{Strayer08botnetdetection}.

Enforcement agencies use this metric to know about efficiency of a botnet. A botnet is evaluated based on it communication efficiency depending on its role(used to forward command and control messages, update bot executable code, gather host-based information)

\paragraph{ Botnet Robustness }

Bots generally lose and gain new members over time. If the victim machines are performing  tasks like storing files for download or sending spam messages from a queue, a higher degree of connection  between bots provides fault tolerance and  recovery ~\cite{Strayer08botnetdetection}.

Enforcement agencies use this metric to know about robustness of a botnet.