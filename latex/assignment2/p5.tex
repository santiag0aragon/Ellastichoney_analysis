Vendors of operating systems should provide tools that provide on-demand and real-time scanning of systems to help remove viruses, spyware, and other malicious software. These valuable applications should be bundled with the operating systems as part of the effort to mitigate the risks from botnet infection attempts. They cannot be a complete replacement to antivirus and antispyware applications used that scan and monitor known viruses and spyware, however they can be a vital resource to provide ongoing protection from a trusted source. An effective and up-to-date software firewall at the OS level can be an effective protection barrier between the users system and the Internet. OS Vendors should provide a sandbox environment to run unverified programs that may contain malicious code downloaded from unverified solution providers, untrusted users and untrusted websites. Software providers that provide email, social networking and instant messaging facilities should enforce strong filtering of content and attachments to mitigate the risk of botnet infections delivering malware that are presumed to be pictures or movies by the end users. Users should be limited to restricted access to applications and OS features, unless all security settings are hardened and the systems are patched with the most current updates.	

Domain registrars can be effective against blocking compromised domains for hosting phishing sites, command and control servers, poisoned and malware delivery sites. To establish every network connection to a particular host, computers have to preform DNS lookup. Domain registrars can use methods such as packet filtering to filter traffic towards banned IPs by blocking access to certain servers from a list of banned or suspected bad domains. Attackers abuse the DNS system to avoid detection by using devious tricks such as fast-flux. In the fast-flux technique, bots continuously register and deregister their IP addresses for a particular domain. Each request by a user to access a malicious site will result in accessing a different bot and possibly different ISP. The only option to combat fast-flux is for the registrar to suspend the bad domain, which has its own positive and negative incentives in doing so. Another proactive approach that registrars can adopt is automated abuse handling and actively responding to external domain suspension requests from financial service providers and other authentic sources of information like intelligence agencies and ISPs.

Most major vendors like Microsoft are providing free tools to remediate the effects of botnet infections, and help users to combat against cyber criminals for their online privacy and security. However, there is a need for more user awareness for better protection and prevention from botnet infections. Once notified by ISPs about infected machines, user must take remedial actions to remove malicious software from their systems. Even when the related malware have been removed, users must take actions and steps to fully recover the systems to a stable state. Remediation is a critical step to curb the impact of bots, though it is recognized that without tools and processes to harden the devices and prevent reinfection, bot infections will repeatedly reoccur.

With ever growing incentives and motives behind spreading infections to run large networks of distributed botnets, the techniques and technologies are changing rapidly with time. Vendors of security services such as antivirus solution providers should put more effort on research and develop solutions to detect, prevent and mitigate the impact of botnet infections. Although the existing practices, tools and techniques to mitigate botnet infections are being tackled to a greater extent compared to the past, however attackers are coming up with clever new ways of exploiting vulnerabilities to spread their networks via botnet infections. Many infections are quite hard to remove, as they may disable windows update, as well as block access to the websites and update servers of antivirus and security software vendors. 
