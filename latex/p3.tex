%!TEX root = main.tex
% \begin{figure}
% \centering
% \includegraphics[width=0.7\linewidth]{"../../../Economics of security/1"}
% \caption{Botnet Resourse requirements and Metrics}
% \label{fig:1}
% \end{figure}
% \begin{figure}
% \centering
% \includegraphics[width=0.7\linewidth]{"../../../Economics of security/2"}
% \label{fig:2}
% \end{figure}


In the previous section we have seen which kind of metrics would be ideal for ISP's and LEA's in facing the botnet infection issue. 

Current research mostly use honeypots to detect where botnet infection traffic is coming from and how it grows over time. In these cases, metrics set out the number of connections/attacks over time to the different IP addresses/longitudinal locations/domain names~\cite{AM2006}.

Other research has been done in differentiating between botnet types by looking at the connection and resources features of different botnet types. After classification, metrics set out the number of connections/attacks over time to the different botnet types~\cite{GJ2007},~\cite{AM2005}.

And last, research has been done in training honeypots to automatically detect botnet infection traffic and metrics are used to show the percentage of correct detections/classifications for th different botnet types~\cite{haltas2014automated}.
