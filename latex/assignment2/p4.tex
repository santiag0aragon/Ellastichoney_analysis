Historically, responses to propagation of botnet-infections within the Internet ecosystem have relied heavily on ISPs, but recently remediation efforts have evolved to include other stake holders / actors to tackle our security issue. These actors have direct involvement with end-user machines to help mitigate the impact of the security issue and have increasingly recognize the long-term impact of malware upon their customers and online services. These actors include: 

\begin{itemize}
\item \textit{Vendors of Operating Systems / Softwares.}Most of the botnet infections are rooted in exploiting vulnerable code in operating system and application software. Since the software market dominates with more characteristics, ease of use, compatibility and features in an application rather than the importance of security thus causing negative externalities. The cost of security gets high due to vendor costs causing reduction in their rewards, therefore more incentives should be provided for them to make cheaper and secure software to compete with free insecure software.
\item \textit{Domain Registrars.}Domain registrars can be effective against blocking compromised domains for hosting phishing sites, command and control servers, poisoned and malware delivery sites.
\item \textit{Vendors of Security Services.}End-users rely heavily on antivirus solutions to secure their systems from threats and vulnerabilities that might harm their privacy and security. Botnet infections provide vendors an ever-growing market for them to sell their products. However, the capabilities of antivirus solutions to mitigate the risks of malware are far less than the expected outcome.
\item \textit{Cloud & Hosting Service Providers.}There is an increasing growth in botnet infections caused by using cloud service providers and web hosting services due to the technological and infrastructural advantages for attackers to use such tools and techniques. It is of prime importance for such service providers to scan and filter traffic used for spreading malware and vulnerable applications.
\item \textit{End-Users.}End-users comprise of home / individuals users and small to medium businesses employees. They create negative externalities caused by risky online behavior and not deploying preventive controls such as antivirus and security software.
\end{itemize}
