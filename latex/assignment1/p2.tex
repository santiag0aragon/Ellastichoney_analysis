%!TEX root = main.tex
\label{cap:sec2}
The security issue defined in section \ref{cap:sec1} have distinct effects depending on the issued party.
% We have identified two identify two parties that might be aimed by the security issue and we arrange them in two groups which are affected in similar ways.
For each party, we identify how the botnet infection (BNI) is negatively influencing their interests, as well as a set of ideal metrics which could help each party to be able to explain and react in a more informed fashion against the BNI issue, and therefore, derive better decisions to take the more effective counter measures to mitigate this problem while maximizing the return on security investment.
\indent
\paragraph{ISP}
For this party it is of main interest to avoid misuse of resources i.e. the ISP would like to block not legitim incoming traffic to its network while the botnet's victim would like to maintain availability of its resources.
% TODO description of each metric, intro to metrics
\begin{itemize}
    \item \textit{BN propagation.} The growth and propagation trend are an ideal tool to quantify and predict the cost of malicious activity within the ISP network.
    \item \textit{Resources (un)availability cost.} The ISP would like to know how much traffic of botnet infection activity is passing through his network. If is able to distinguish botnet activity from user activity he could measure how much resources are being misused and estimate the cost of having botnet inside his network and therefore justify the investment on security countermeasures.
    \item \textit{Network infection rate.} The ISP could quantify his subnet infection rate prevention and reaction policies would be better targeted, i.e., if the subnet x.x.x.0 is heavy infected the ISP could target only this portion his network.
    % \item What is honey to attacker
\end{itemize}
\indent
\paragraph{LEA}
This party is interested in prevent, prosecute and punish the entity behind a security issue, i.e., to prevent BNI is of main interest to transfer the risk from the BN victims to the ISP and/or to encourage the zombies machines to increase their security protection, while for prosecution and punishment it is important to have mechanisms to understand and hunt down the BN.
\begin{itemize}
    % We need more ideal metrics
    \item \textit{BN geographical and digital location.} With geographical and digital location metrics the interested party could measure the infected population in a particular region either in the  physical or digital world, i.e. how many infected machines are in the south of Chine, or how many infected machine does a ISP have.
    \item \textit{BNI target machines.} By characterizing which kind of machines are the weakest link in a network, the LEA could launch prevention campaigns describing which kind of machines are more probable to become part of a botnet, i.e., operative system version, browser, etc.
\end{itemize}



