
The security issue defined in section \ref{sec1} have distinct effects depending on the issued party. We identify several parties that might be aimed by the security issue and we arrange them in two groups which are affected in similar ways. For each group, we identify how the botnet (BN) is negatively influencing their interests, as well as a set of ideal metrics which could help each party to derive better decisions to take the more effective counter measures to mitigate their problem while maximizing the return on security investment.

\indent
\paragraph{Internet Service Provider (ISP) \& Botnet's victim}
For this parties it is of main interest to avoid misuse of resources i.e. the ISP would like to block not legitim incoming traffic to its network while the botnet's victim would like to maintain availability of its resources.
% TODO description of each metric, intro to metrics
\paragraph{Metrics}
\begin{itemize}
    \item Attack prediction.
    \item BN identity detection.
    \item Resources (un)availability cost.
    % \item What is honey to attacker
\end{itemize}
\indent
% TODO: Botnet's zombie machine may be irrelevant to our study
% \paragraph{Botnet's zombie machine}
% Even when this party might not be directly interested in avoiding being a tool of the botnet, it should be interested in not misusing its computational resources as part of the botnet.
% \paragraph{Metrics}
% \begin{itemize}
%     \item Rootkit detection.
%     \item Resources misuse cost.
% \end{itemize}
\indent
\paragraph{Regulator authority \& Law enforcement agencies}
The parties interested in prevent, prosecute and punish the ones behind a security issue, i.e., For the regulator authority is of main interest to transfer the risk to the ISP, to encourage the zombies machines to increase their security protection, while for a law enforcement agency is to have mechanisms to hunt down the attacker
\paragraph{Metrics}
\begin{itemize}
    % We need more ideal metrics
    \item BN location detection
\end{itemize}


