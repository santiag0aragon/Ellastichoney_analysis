%!TEX root = main.tex

% What risk strategies can the problem owner follow to reduce the security issue as measured in your first assignment?
In this section we describe possible risk strategies that the ISP can follow in order to reduce the security impact of botnet evolution.
\paragraph{To work with Industry and other stakeholders on developing a policy to implement}


\paragraph{Defense against attack by Bot}
Security must be built-in during each phase of the system development and this can be done by using Intrusion Prevention System (IPS), Transport Layer Security (TLS)/Secure Sockets Layer (SSL) protocol and Correct coding of Web service/applications (without security flaws which makes it resistant to threats)~\cite{stankovic2009defense}.


\paragraph{Monitoring, detection and studying of Bot}
All users of computers and the system administrators should detect presence and/or activities of Botnet and prevent it from influencing upon the computers by constant following of the activities on the computers such as monitoring log files, detecting threats and finding counter measures~\cite{stankovic2009defense}.

\begin{itemize}
    \item a registration component
    \item an awareness-raising component
    \item guidance on network management
    \item high-level advice on how to respond to threats
    \item a reporting component
\end{itemize}

Attempting to instruct users about how to shield themselves from installing malware and inadvertently transforming their PCs into bots is a key component to raise the level of security awareness~\cite{stankovic2009defense}.


\paragraph{To develop legislative punisment policy}
ISPs are in a better position to understand the issues associated with botnets and can act on the botnet threat. They stand a better chance at formulating the solution (legislative measures) along with legislative-punishment body to prevent the attackers from trying to carry the attacks~\cite{stankovic2009defense}.


Policy makers and ISPs must consider how best to implement authentication mechanisms that encourage reliable communications between ISPs, consumers and other actors.
\paragraph{The Addressing Layer}
This involves strategies targeting the routing and the addressing layer of a botnet infrastructure. Bots in the local network cannot contact the original C\&C server when intervened in addressing which usually take place in two steps.


In the first step, a site administrator can control the local DNS resolver(which handles the DNS requests forwards the request to an authoritative DNS server) and instruct to return a specially crafted response to specific queries. In the next step, local routers can be equipped with routing table entries to sinkhole certain addresses or redirect them to different hosts~\cite{leder2009proactive}.


\paragraph{The Command Layer}
This involves attacking the command layer of a botnet with the knowledge of the protocol used. An easy example would be an IRC-based network where a command like remove can instruct bots to uninstall themselves from infected systems~\cite{leder2009proactive}.


\paragraph{Exploitation}
Exploit based strategies make use of presence of bugs and programming flaws in bots that result in vulnerabilities which can be exploited to gain control either over a central component or over infected machines. An examples of such vulnerabilities would be security holes in software or remotely-exploitable buffer overflows~\cite{leder2009proactive}.


