%!TEX root = main.tex
\label{cap:sec1}

It is impractical and very expensive to defend against every possible vulnerability~\cite{TARA}. Therefore, decision-makers must design and implement effective security strategies about where and when to invest time and money~\cite{ROSI}. To make decisions on security investments, models are used and these models are build on security metrics~\cite{BR2010}. Since the definition of these metrics are the basis of a good security model and therefore, they are a key element to produce valuable information for the decision-makers. For example, not all vulnerabilities will be attacked, thus knowing the most likely attack vectors is key in planning effective security measurements~\cite{TARA}.


One way of gathering attacker information is by using \textit{honeypots}. Honeypots are isolated and monitored systems that emulate to be vulnerable in order to attract attackers and track their behavior~\cite{WP2010}, i.e., attack attempt data could be used to define metrics that capture different aspects of attacker behavior. In particular we take a closer look to \textit{Elastichoney}~\cite{BR2010} project. It emulate to be a elasticsearch server suffering of a remote code execution (RCE) vulnerability identified as CVE- 2015-1427. This vulnerability allow attackers to execute Java based code by querying the server~\cite{CVE}. Furthermore, the Elastichoney project have gathered data over two months and tracked about 8k attempts to attack from over 300 unique IP addresses~\cite{BR2010}.

In Elastichoney logs the major security issue that can be found is \textit{Botnet Infection} attempts, i.e. when a botnet machine is trying propagate the botnet by infecting another machine.
% The attacking behavior of botnet infection found in the data can be summarised as follows:
% \begin{itemize}
% \item[-] Using the code execution vulnerability, operation steps were performed on multiple different files: \textit{wget} file, \textit{chmod 777} on file location, execute file, removing file.
% \item[-] Operations on the same file were requested by multiple different sources.
% \end{itemize}

In this work we take a look to two different parties who have great incentive to know about the attacking behavior of botnet infection, namely \textit{Internet Service Providers (ISP's)} and \textit{Law enforcement agencies (LEA's)}.

Botnet infection is a major security issue for ISP's, because 27\% of the overall unwanted traffic on the Internet can be attributed to botnet-related spreading activity~\cite{AM2006}. And this is a great loss in productivity for ISP's. On the other hand, LEA's would counter this security issue, because botnets are heavily used as a platform for various criminal business models like sending spam, committing click fraud, harvesting account credentials, launching denial-of-service attacks, installing scareware and phishing~\cite{AR2013}.

In the rest of this paper we discuss ideal and state-of-the-art metrics that measure different aspects of botnet infection and how these metrics can be used by ISP's and enforcement agencies to make effective security investments. Finally, we propose, implement compare and evaluate our own metrics.

