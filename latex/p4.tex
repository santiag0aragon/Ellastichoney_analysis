%!TEX root = main.tex



In this section, we propose a number of metrics that can be used to extract valuable information from the \textit{Ellastichoney} dataset. For each metric we define its input, output and the utility for the interested party mentioned in section \ref{cap:sec2}.
\indent
\subsection{BN propagation}
    % \paragraph{Number of Attacks/Recognition:}
    % This metric classify  and quantify the attacker's attempt either as an \textit{attack}, attempt at actually doing something bad or a \textit{recognition}, when an attacker is scanning the network to see if some instances are vulnerable.

    \paragraph{BN growth}It measures the number new IPs where the BNI attempts are being performed. It is useful to see the aggregate growth of the BN, and to help to infer a BNI rate.

    \paragraph{BNI attempts}
    It measures the number of attacks executed daily. We will use this data to analyze the BN activity. This metric could be useful to measure the effectiveness of mitigating controls.

\subsection{Network infection rate}
    \paragraph{Subnet infection rate} By summarizing the number of attacks that a particular IP perform and listing all the IP within a network, we can assign certain subnetworks an infection rate. I.e. Number infected IPs vs total IPs in a certain subnetwork.

    We can view the top source IPs that were most prominent during the attack, simply by analyzing the number of attacks coming from each IP. This could be used for reputation analysis and to identify of infected IPs, i.e. a LEA could publish a report of the most active IP within a BN to encourage the ISPs to take action against them.

    \paragraph{BNI IP source}
    Taking a look to top source IPs that were most active, i.e. counting the number of attacks coming from each IP and sorting them according its activity. This could be used for reputation analysis and to identify of infected IPs, i.e. a LEA could publish a report of the most active IP within a BN to encourage the ISPs to take action against them.


\subsection{BN geographical and digital location.} The input for this metric will include the number of attacks coming from various counties and also pinpointing the regional locations according to cities. This metric will help measure the scale of attacks coming from a particular geographical location. Furthermore, we can assess to block the IP ranges as per the geographical source of the attack coming from to mitigate the risk of potential impacts to critical infrastructure.


\subsection{BNI target machines.}
    We identify the characteristics of the BN machines such as operating systems, web browsers and the content type used to exploit the vulnerability. The output could help the ISP to target awareness campaigns across its subscribers.
\indent

